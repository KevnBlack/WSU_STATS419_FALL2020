\documentclass[]{article}
\usepackage[left=1in,top=1in,right=1in,bottom=1in]{geometry}


%%%% more monte %%%%
% thispagestyle{empty}
% https://stackoverflow.com/questions/2166557/how-to-hide-the-page-number-in-latex-on-first-page-of-a-chapter
\usepackage{color}
% \usepackage[table]{xcolor} % are they using color?

% \definecolor{WSU.crimson}{HTML}{981e32}
% \definecolor{WSU.gray}{HTML}{5e6a71}

% \definecolor{shadecolor}{RGB}{248,248,248}
\definecolor{WSU.crimson}{RGB}{152,30,50} % use http://colors.mshaffer.com to convert from 981e32
\definecolor{WSU.gray}{RGB}{94,106,113}

%%%%%%%%%%%%%%%%%%%%%%%%%%%%

\newcommand*{\authorfont}{\fontfamily{phv}\selectfont}
\usepackage{lmodern}


  \usepackage[T1]{fontenc}
  \usepackage[utf8]{inputenc}




\usepackage{abstract}
\renewcommand{\abstractname}{}    % clear the title
\renewcommand{\absnamepos}{empty} % originally center

\renewenvironment{abstract}
 {{%
    \setlength{\leftmargin}{0mm}
    \setlength{\rightmargin}{\leftmargin}%
  }%
  \relax}
 {\endlist}

\makeatletter
\def\@maketitle{%
  \pagestyle{empty}
  \newpage
%  \null
%  \vskip 2em%
%  \begin{center}%
  \let \footnote \thanks
    {\fontsize{18}{20}\selectfont\raggedright  \setlength{\parindent}{0pt} \@title \par}%
}
%\fi
\makeatother






\usepackage{color}
\usepackage{fancyvrb}
\newcommand{\VerbBar}{|}
\newcommand{\VERB}{\Verb[commandchars=\\\{\}]}
\DefineVerbatimEnvironment{Highlighting}{Verbatim}{commandchars=\\\{\}}
% Add ',fontsize=\small' for more characters per line
\usepackage{framed}
\definecolor{shadecolor}{RGB}{248,248,248}
\newenvironment{Shaded}{\begin{snugshade}}{\end{snugshade}}
\newcommand{\AlertTok}[1]{\textcolor[rgb]{0.94,0.16,0.16}{#1}}
\newcommand{\AnnotationTok}[1]{\textcolor[rgb]{0.56,0.35,0.01}{\textbf{\textit{#1}}}}
\newcommand{\AttributeTok}[1]{\textcolor[rgb]{0.77,0.63,0.00}{#1}}
\newcommand{\BaseNTok}[1]{\textcolor[rgb]{0.00,0.00,0.81}{#1}}
\newcommand{\BuiltInTok}[1]{#1}
\newcommand{\CharTok}[1]{\textcolor[rgb]{0.31,0.60,0.02}{#1}}
\newcommand{\CommentTok}[1]{\textcolor[rgb]{0.56,0.35,0.01}{\textit{#1}}}
\newcommand{\CommentVarTok}[1]{\textcolor[rgb]{0.56,0.35,0.01}{\textbf{\textit{#1}}}}
\newcommand{\ConstantTok}[1]{\textcolor[rgb]{0.00,0.00,0.00}{#1}}
\newcommand{\ControlFlowTok}[1]{\textcolor[rgb]{0.13,0.29,0.53}{\textbf{#1}}}
\newcommand{\DataTypeTok}[1]{\textcolor[rgb]{0.13,0.29,0.53}{#1}}
\newcommand{\DecValTok}[1]{\textcolor[rgb]{0.00,0.00,0.81}{#1}}
\newcommand{\DocumentationTok}[1]{\textcolor[rgb]{0.56,0.35,0.01}{\textbf{\textit{#1}}}}
\newcommand{\ErrorTok}[1]{\textcolor[rgb]{0.64,0.00,0.00}{\textbf{#1}}}
\newcommand{\ExtensionTok}[1]{#1}
\newcommand{\FloatTok}[1]{\textcolor[rgb]{0.00,0.00,0.81}{#1}}
\newcommand{\FunctionTok}[1]{\textcolor[rgb]{0.00,0.00,0.00}{#1}}
\newcommand{\ImportTok}[1]{#1}
\newcommand{\InformationTok}[1]{\textcolor[rgb]{0.56,0.35,0.01}{\textbf{\textit{#1}}}}
\newcommand{\KeywordTok}[1]{\textcolor[rgb]{0.13,0.29,0.53}{\textbf{#1}}}
\newcommand{\NormalTok}[1]{#1}
\newcommand{\OperatorTok}[1]{\textcolor[rgb]{0.81,0.36,0.00}{\textbf{#1}}}
\newcommand{\OtherTok}[1]{\textcolor[rgb]{0.56,0.35,0.01}{#1}}
\newcommand{\PreprocessorTok}[1]{\textcolor[rgb]{0.56,0.35,0.01}{\textit{#1}}}
\newcommand{\RegionMarkerTok}[1]{#1}
\newcommand{\SpecialCharTok}[1]{\textcolor[rgb]{0.00,0.00,0.00}{#1}}
\newcommand{\SpecialStringTok}[1]{\textcolor[rgb]{0.31,0.60,0.02}{#1}}
\newcommand{\StringTok}[1]{\textcolor[rgb]{0.31,0.60,0.02}{#1}}
\newcommand{\VariableTok}[1]{\textcolor[rgb]{0.00,0.00,0.00}{#1}}
\newcommand{\VerbatimStringTok}[1]{\textcolor[rgb]{0.31,0.60,0.02}{#1}}
\newcommand{\WarningTok}[1]{\textcolor[rgb]{0.56,0.35,0.01}{\textbf{\textit{#1}}}}



\title{\textbf{\textcolor{WSU.crimson}{The Influence of Biological Sex
on Human Body Part Ratios}} \newline \textbf{\textcolor{WSU.gray}{How
these Ratios Compare Between the Sexes}}  }
 

%  

% \author{ \Large true \hfill \normalsize \emph{} }
\author{\Large Kevin A. Black -
\href{mailto:kevin.black@wsu.edu}{\nolinkurl{kevin.black@wsu.edu}}\vspace{0.05in} \newline\normalsize\emph{Washington
State University Vancouver}  }


\date{November 04, 2020}
\setcounter{secnumdepth}{3}

\usepackage{titlesec}
% See the link above: KOMA classes are not compatible with titlesec any more. Sorry.
% https://github.com/jbezos/titlesec/issues/11
\titleformat*{\section}{\bfseries}
\titleformat*{\subsection}{\bfseries\itshape}
\titleformat*{\subsubsection}{\itshape}
\titleformat*{\paragraph}{\itshape}
\titleformat*{\subparagraph}{\itshape}

% https://code.usgs.gov/usgs/norock/irvine_k/ip-092225/


%\titleformat*{\section}{\normalsize\bfseries}
%\titleformat*{\subsection}{\normalsize\itshape}
%\titleformat*{\subsubsection}{\normalsize\itshape}
%\titleformat*{\paragraph}{\normalsize\itshape}
%\titleformat*{\subparagraph}{\normalsize\itshape}

% https://tex.stackexchange.com/questions/233866/one-column-multicol-environment#233904
\usepackage{environ}
\NewEnviron{auxmulticols}[1]{%
  \ifnum#1<2\relax% Fewer than 2 columns
    %\vspace{-\baselineskip}% Possible vertical correction
    \BODY
  \else% More than 1 column
    \begin{multicols}{#1}
      \BODY
    \end{multicols}%
  \fi
}





\usepackage{natbib}
\setcitestyle{aysep={}} %% no year, comma just year
% \usepackage[numbers]{natbib}
\bibliographystyle{./../biblio/ormsv080.bst}



\usepackage[strings]{underscore} % protect underscores in most circumstances




\newtheorem{hypothesis}{Hypothesis}
\usepackage{setspace}


%%%%%%%%%%%%%%%%%%%%%%%%%%%%%%%%%%%%%%%%%%%%%%%%%%%%%
%%% MONTE ADDS %%%

\usepackage{fancyhdr} % fancy header 
\usepackage{lastpage} % last page 

\usepackage{multicol}


\usepackage{etoolbox}
\AtBeginEnvironment{quote}{\singlespacing\small}
% https://tex.stackexchange.com/questions/325695/how-to-style-blockquote


\usepackage{soul}			%% allows strike-through
\usepackage{url}			%% fixes underscores in urls
\usepackage{csquotes}		%% allows \textquote in references
\usepackage{rotating}		%% allows table and box rotation
\usepackage{caption}		%% customize caption information
\usepackage{booktabs}		%% enhance table/tabular environment
\usepackage{tabularx}		%% width attributes updates tabular
\usepackage{enumerate}		%% special item environment
\usepackage{enumitem}		%% special item environment

\usepackage{lineno}		%% allows linenumbers for editing using \linenumbers
\usepackage{hanging}


\usepackage{mathtools}  	%% also loads amsmath
\usepackage{bm}		%% bold-math
\usepackage{scalerel}	%% scale one element (make one beta bigger font)

\newcommand{\gFrac}[2]{ \genfrac{}{}{0pt}{1}{{#1}}{#2} }

\newcommand{\betaSH}[3]{  \gFrac{\text{\tiny #1}}{{\text{\tiny #2}}}\hat{\beta}_{\text{#3}}   }
\newcommand{\betaSB}[3]{              ^{\text{#1}} _{\text{#2}} \bm{\beta} _{\text{#3}}                   }  %% bold
\newcommand{\bigEQ}{  \scaleobj{1.5}{{\ }= } }
\newcommand{\bigP}[1]{  \scaleobj{1.5}{#1 } }





\usepackage{endnotes}  % he already does this ...
\renewcommand{\enotesize}{\normalsize}
% https://tex.stackexchange.com/questions/99984/endnotes-do-not-be-superscript-and-add-a-space
\renewcommand\makeenmark{\textsuperscript{[\theenmark]}} % in brackets %
% https://tex.stackexchange.com/questions/31574/how-to-control-the-indent-in-endnotes
\patchcmd{\enoteformat}{1.8em}{0pt}{}{}

\patchcmd{\theendnotes}
  {\makeatletter}
  {\makeatletter\renewcommand\makeenmark{\textbf{[\theenmark]} }}
  {}{}



% https://tex.stackexchange.com/questions/141906/configuring-footnote-position-and-spacing

\addtolength{\footnotesep}{5mm} % change to 1mm

\renewcommand{\thefootnote}{\textbf{\arabic{footnote}}}
\let\footnote=\endnote
%\renewcommand*{\theendnote}{\alph{endnote}}
%\renewcommand{\theendnote}{\textbf{\arabic{endnote}}}


\renewcommand*{\notesname}{ENDNOTES}

\makeatletter
\def\enoteheading{\section*{\notesname
  \@mkboth{\MakeUppercase{\notesname}}{\MakeUppercase{\notesname}}}%
  \mbox{}\par\vskip-2.3\baselineskip\noindent\rule{.5\textwidth}{0.4pt}\par\vskip\baselineskip}
\makeatother


\renewcommand*{\contentsname}{TABLE OF CONTENTS}

\renewcommand*{\refname}{REFERENCES}


%\usepackage{subfigure}
\usepackage{subcaption}

\captionsetup{labelfont=bf}  % Make Table / Figure bold

%%% you could add elements here ... monte says .... %%%
%\usepackage{mypackageForCapitalH}


%%%%%%%%%%%%%%%%%%%%%%%%%%%%%%%%%%%%%%%%%%%%%%%%%%%%%

% set default figure placement to htbp
\makeatletter
\def\fps@figure{htbp}
\makeatother

\usepackage{hyperref}

% move the hyperref stuff down here, after header-includes, to allow for - \usepackage{hyperref}

\makeatletter
\@ifpackageloaded{hyperref}{}{%
\ifxetex
  \PassOptionsToPackage{hyphens}{url}\usepackage[setpagesize=false, % page size defined by xetex
              unicode=false, % unicode breaks when used with xetex
              xetex]{hyperref}
\else
  \PassOptionsToPackage{hyphens}{url}\usepackage[draft,unicode=true]{hyperref}
\fi
}

\@ifpackageloaded{color}{
    \PassOptionsToPackage{usenames,dvipsnames}{color}
}{%
    \usepackage[usenames,dvipsnames]{color}
}
\makeatother
\hypersetup{breaklinks=true,
            bookmarks=true,
            pdfauthor={Kevin A. Black -
\href{mailto:kevin.black@wsu.edu}{\nolinkurl{kevin.black@wsu.edu}} (Washington
State University Vancouver)},
             pdfkeywords = {Shapiro-Wilk test, Pearson Product-Moment
Correlation, null hypothesis, alternative hypothesis, two-sample t-test,
data provenance.},  
            pdftitle={The Influence of Biological Sex on Human Body Part
Ratios: How these Ratios Compare Between the Sexes},
            colorlinks=true,
            citecolor=blue,
            urlcolor=blue,
            linkcolor=magenta,
            pdfborder={0 0 0}}
\urlstyle{same}  % don't use monospace font for urls

% Add an option for endnotes. -----

%
% add tightlist ----------
\providecommand{\tightlist}{%
\setlength{\itemsep}{0pt}\setlength{\parskip}{0pt}}

% add some other packages ----------

% \usepackage{multicol}
% This should regulate where figures float
% See: https://tex.stackexchange.com/questions/2275/keeping-tables-figures-close-to-where-they-are-mentioned
\usepackage[section]{placeins}



\pagestyle{fancy}   
\lhead{\textcolor{WSU.crimson}{\textbf{ The Influence of Biological Sex
on Human Body Part Ratios }}}
\chead{}
\rhead{\textcolor{WSU.gray}{\textbf{  Page\ \thepage\ of\ \protect\pageref{LastPage} }}}
\lfoot{}
\cfoot{}
\rfoot{}


\begin{document}
	
% \pagenumbering{arabic}% resets `page` counter to 1 
%    

% \maketitle

{% \usefont{T1}{pnc}{m}{n}
\setlength{\parindent}{0pt}
\thispagestyle{plain}
{\fontsize{18}{20}\selectfont\raggedright 
\maketitle  % title \par  

}

{
   \vskip 13.5pt\relax \normalsize\fontsize{11}{12} 
   
\textbf{\authorfont Kevin A. Black -
\href{mailto:kevin.black@wsu.edu}{\nolinkurl{kevin.black@wsu.edu}}} \hskip 15pt \emph{\small Washington
State University Vancouver}   

}

}








\begin{abstract}

    \hbox{\vrule height .2pt width 39.14pc}

    \vskip 8.5pt % \small 

\noindent Throughout history, there have been evident traits of
uniqueness among the multicellular organisms that roam the Earth. A
common theme among these creatures relates to the differences in
biological sexes, especially when it comes to size or behavior. In the
endeavor to establish some distinct patterns between the biological
sexes of humans, male and female, students of the STAT 419 course
(Introduction to Multivariate Statistics) at Washington State University
conducted a large-scale survey amongst each other and their peers. The
students recorded the data in a consistent format, which was then
compiled by the instructor of the class Monte Shaffer. After the
compilation of data, students were free to formulate their own questions
and look for recognizable patterns amongst the values. These students
also practiced data provenance whilst conducting their research to make
sure the body measurement data wasn't misused in the process.

\vspace{0.25cm}

For this research paper in particular, I address the topic of how
biological sex influences the ratios of particular body parts.
Specifically, this paper looks at the influence that being male or
female has on (1) the ratio of height and arm span and (2) the ratio of
height and head height. The steps taken in this research involved
performing a Pearson correlation test on the ratio between height and
arm span, while a two-sample t-test was performed on the ratio between
height and head height. The results of these tests found that while
there was a noticeable variation in the measurements of male and female
body parts, the previously mentioned ratios were very similar, which
fall in line with existing body ratio/proportion knowledge about
biological males and females.


\vskip 8.5pt \noindent \textbf{\underline{Keywords}:} Shapiro-Wilk test,
Pearson Product-Moment Correlation, null hypothesis, alternative
hypothesis, two-sample t-test, data provenance. \par

    




    
    \hbox{\vrule height .2pt width 39.14pc}
    \vskip 5pt 
    \hfill \textbf{\textcolor{WSU.gray}{ November 04, 2020 } }
    \vskip 5pt 
    
\end{abstract}


\vskip -8.5pt



 % removetitleabstract

\noindent  

\section{Introduction}
\label{sec:intro}

As a customary portion of the STAT 419 course at Washington State
University, the students involved were required to take part in a data
collection/manipulation project involving the measurements of various
body parts. Following the data collection, professor Monte Shaffer
compiled all contributions made by the students into a single,
pipe-delimited text file whose explicit content is to remain
confidential and only used for the sake of addressing a variety of
research questions.

\vspace{0.25cm}

After receiving the data set (\(n=428\)) and cleaning up particular
observations using a range of methodologies based on what I had in mind
for how this data will be used, I determined that I would focus on what
kind of influence biological sex has on certain human body part ratios.
To supplement my inquiry of this data, I centered my research questions
around particular parts of the human body, which would be \(height\),
\(arm.span\), and \(head.height\) of the data set. These variables
correspond to the standing height of the individual with no shoes on,
the length from one middle finger to the other with fully extended arms,
and the height from the top of the head to below the chin, respectively.
Research questions pertaining to this study begins in section
\ref{sec:rq} and supportive R code can be found in section
\ref{sec:r-setup}.

\section{Data Description}
\label{sec:data}

As mentioned in the introduction, the data set utilized throughout this
research paper was supplied through the combined efforts of professor
Monte Shaffer and the students of the STAT 419 class at Washington State
University. Each student was tasked with recording distinct observations
of 37 attributes from 10 different people, ideally with an even mix of
males and females. For each observation (person), there are measurements
of body parts, data collector/respondent identifiers ran through a MD5
hash function, and general information about each respondent. Body part
measurements were recorded using body measuring tape in either inches or
centimeters, but for the sake of consistency throughout the research
paper, all centimeter values were converted to inches and will be
treated as inches from here on.

\vspace{0.25cm}

These observations were recorded in early September 2020 and compiled by
the instructor for our use in late October 2020. Given that these
observations were recorded amid the COVID-19 pandemic, each student was
required to make a simple, yet descriptive handout that would detail how
one would go about recording their own body measurements and the
necessary values to take note of. This would be an ideal situation of
how observations were recorded and sent electronically by each
respondent, but observations could also be taken in person given that
the surveyor and respondent were comfortable being in close proximity of
each other. An example of a two-page handout created by Kevin Black, as
well as further information regarding the attributes of this data set,
can be found in Appendix \ref{sec:appendix-data-handout} and Appendix
\ref{sec:appendix-dataset-ex}, respectively.

\vspace{0.25cm}

On the surface, the purpose for writing this research paper and
collecting the necessary data can simply be attributed to project
requirements for a university course. However, the deeper reasoning
behind why this research paper was composed the first place was to give
students a more thorough understanding of the data analytics process.
More specifically, how to not just work with the data, but how to
understand the data and derive effective questions, how to test for
patterns and conclusions in a statistical, analytic environment, and how
to exercise data provenance practices. This process will be very similar
between all data focused projects in a data analyst's career, so this is
a good starting point for garnering crucial experience.

\section{How does being male or female influence the ratios between certain body parts?}
\label{sec:rq}

Making particular deductions throughout history regarding body
measurements among males and females is more than likely due to the
distinct patterns one can find when comparing the biological sexes.
Those born as males or females have about as many differing features to
one another than their similar features, both internally and externally.
Internally speaking, men typically have deeper voices, a faster
metabolism, and can easily build muscle mass, whereas women possess a
much more complicated reproductive system, have the ability to
breastfeed, and live longer on average \citep{Wolchover:2011}. What
about externally? How does being male or female influence the ratios
between certain body parts? The latter primary question will be
elaborated through the following sub-questions, focusing on particular
ratios that are likely to yield some insight on the different external
measurements between males and females.

\subsection{How does 'height' compare to 'arm span' between males and females?}
\label{sec:rq2}

The body part ratio between \(height\) and \(arm.span\) is a well-known
one where ``for most people, their arm span is about equal to their
height. Mathematicians say the arm span to height ratio is one to one''
\citep{Brabandere:2017}. To set up this sub-question, let's first
declare some null and alternative hypotheses for males and females. For
males, \(H_0:\rho_m=0\) and \(H_1:\rho_m\neq0\). For females,
\(H_0:\rho_f=0\) and \(H_1:\rho_f\neq0\). We must also check both
vectors for normality prior to correlation testing by performing a
Shapiro-Wilk test with the \texttt{shapiro.test()} function. Running
this function gives results for males \(height\) (\(W=0.73255\),
\(p=2.303\times10^{-11}\)) and \(arm.span\) (\(W=0.80943\),
\(p=2.664\times10^{-9}\)), as well as for females \(height\)
(\(W=0.78534\), \(p=6.762\times10^{-11}\)) and \(arm.span\)
(\(W=0.82018\), \(p=8.305\times10^{-10}\)). All of these test statistics
show that we can move forward with the \texttt{cor.test()} function as
each of these tested populations may come from normal distributions. The
correlation test will be conducted using the method of Pearson
Product-Moment Correlation.

\vspace{0.25cm}

Given the results of \texttt{cor.test()}, when it comes to males we
reject our null hypothesis \(H_0:\rho_m=0\) and accept our alternative
hypothesis \(H_1:\rho_m\neq0\) as there is some clear indication of
strong, positive correlation among the male body parts
(\(t_{86}=33.105\), \(p<2.2\times10^{-16}\)). The same conclusion can be
made with females, where we reject our null hypothesis \(H_0:\rho_f=0\)
and accept our alternative hypothesis \(H_1:\rho_f\neq0\)
(\(t_{100}=40.488\), \(p<2.2\times10^{-16}\)). Additionally, based
solely on figure \ref{fig:sq1} below that was generated from the R code
in section \ref{sec:first-subq}, we can see that there are strong
positive correlations for both males and females with regard to their
respective \(height\) and \(arm.span\) values.

\begin{figure}[!ht]
    \begin{center}
        \scalebox{1.00}{    \includegraphics[trim = 0 0 0 0,clip,width=0.85\textwidth]{figures/sq1.png} }
        \caption{Plots/correlation values for males and females regarding height and arm span.}
        \label{fig:sq1}
    \end{center}
\end{figure}

\subsection{How many average 'head height' lengths are males and females relative to their respective, average 'height'?}
\label{sec:rq4}

To form the foundation for how we'll go about answering this third
sub-question, let's consider the population means for \(head.height\)
and \(height\) for males and females. To test the two population means
for equality between males and females, we will utilize a two-sample
t-test. The null hypotheses and alternative hypotheses can be set up as
follows:

\begin{itemize}
\item
  Regarding \(head.height\), to determine if the population means for
  males and females are equal, let the null hypothesis be
  \(H_0: \mu_{m.head}=\mu_{f.head}\) and the alternative hypothesis be
  \(H_1: \mu_{m.head} \neq \mu_{f.head}\).
\item
  Regarding \(height\), to determine if the population means for males
  and females are equal, let the null hypothesis be
  \(H_0: \mu_{m.height}=\mu_{f.height}\) and the alternative hypothesis
  be \(H_1: \mu_{m.height} \neq \mu_{f.height}\).
\end{itemize}

Based on the two-sample t-test results generated by the code in section
\ref{sec:second-subq}, regarding the comparison of mean \(head.height\)
between males and females, we can reject our null hypothesis
\(H_0: \mu_{m.head}=\mu_{f.head}\) and accept our null hypothesis
\(H_1: \mu_{m.head} \neq \mu_{f.head}\) as there is evidence of at least
one statistical difference in the mean head heights between the sexes
(\(t_{182.64}=3.8231\), \(p=1.807\times10^{-4}\)). And as for
\(height\), we can make a similar conclusion where we reject our null
hypothesis \(H_0: \mu_{m.height}=\mu_{f.height}\) and accept our null
hypothesis \(H_1: \mu_{m.height} \neq \mu_{f.height}\) as there is
evidence of at least one statistical difference in the mean heights
between the sexes (\(t_{184.65}=3.9891\), \(p=9.552\times10^{-5}\)).

\vspace{0.25cm}

Now that it's established that there are indeed statistical differences
between males and females when it comes to their respective mean
\(head.height\) and \(height\) values, we can calculate the explicit
values to determine how many average \(head.height\) lengths males and
females are relative to their respective, average \(height\). From the
results generated by the code in section \ref{sec:second-subq}, we can
see that the average head height for males is 18.19722 inches and for
females is 14.31103 inches, whereas the average height for males is
138.7041 inches and for females is 107.5450 inches. When measured in
terms of head heights, this leads to values of 7.622273 heads for males
and 7.514837 heads for females.

\newpage

\section{Key Findings}
\label{sec:findings}

For the first sub-question, the correlation between \(height\) and
\(arm.span\) measured in at 0.96293 for males and ever-so-slightly
higher at 0.97083 for females. These values mean that the heights and
arm spans for both sexes change at almost the same rate as one another,
indicating that there is little to no difference between males and
females for this common body ratio. The only distinction between the
sexes in this sample is that they operate on different value ranges,
where height and arm span are larger for males than for females. As for
the second sub-question, the surprisingly about-equal results of
7.622273 heads for males and 7.514837 heads for females lines up with
common proportion knowledge about human height in terms of head height,
where ``the average adult human is technically seven-and-one-half heads
tall\ldots{} the average adult female is smaller than the average adult
male, however you'll notice that they are both proportionately similar''
\citep{Larson:2014}.

\section{Conclusion}
\label{sec:conclusion}

It's important to reiterate that there's no doubt that males and females
can have quite stark differences when it comes to the measurements of
external features, and the research conducted above shows that while
there can be some differences, there are also some notable similarities.
For the correlation values regarding these particular samples, it would
appear that males and females are almost equally as likely to have a
height matching their arm span. This shows that males and females can
have very similar body part ratios despite having varied measurements.
The research conducted for this sub-question is simply a supplement to
information that has already been known throughout history and simply
reaffirms the existing beliefs for this specific ratio of body parts.
Lastly, for the heights of males and females measured in terms of their
respective head heights, they once again both have very similar ratios
despite them having very different average, respective values of
\(height\) and \(head.height\). The evidence provided shows that even
though males and females can have varying body measurements between each
other as a group, the influence of biological sex on human body part
ratios is seemingly non-existent in this case as the ratios were similar
in both studies.

\newpage

\section{APPENDICES}
\label{sec:appendix}

\subsection{Data Provenance}
\label{sec:appendix-data-provenance}

\subsubsection{Utilization of Data Provenance}
\label{sec:appendix-provenance-explained}

While it could be seen as an application mainly used in large-scale data
analytics projects, the multitude of steps to practice data provenance
was also used to handle the vulnerable data used in this research.
Auxiliary files such as the R project files can be found on the
\href{https://github.com/KevnBlack/WSU_STATS419_FALL2020/tree/master/PROJECT-01}{project repository}
through GitHub, while the specific functions used in this report can be
found in the \texttt{functions-project-measure.R}
\href{https://raw.githubusercontent.com/KevnBlack/WSU_STATS419_FALL2020/master/functions/functions-project-measure.R}{file}
or in section \ref{sec:functions}. The data set itself was not saved to
any location online due to privacy concerns, but it was organized and
saved onto a local hard drive for immediate use.

\vspace{0.25cm}

The collection and general organization process for the data set is
thoroughly described in sections \ref{sec:data} and
\ref{sec:appendix-dataset-ex}. Cleaning for the actual substance of the
data set was performed through the use of the
\texttt{prepareMeasureData()} function, where the data set was
manipulated to more better fit the aims of addressing the particular
research questions for this report. Cleaning this data set involved
multiple steps, such as: omitting rows containing NA values based solely
on the body measurement columns, setting a consistent naming convention
for gender and units, converting all body measurement values to inches,
and scaling the body measurement data. Keeping the research questions in
mind and despite the cleaning of all columns, the only body measurement
fields utilized were \(height\), \(arm.span\), \(foot.length\),
\(elbow.armpit\), and \(head.height\).

\vspace{0.25cm}

Prior to \texttt{prepareMeasureData()}, the data is imported via the
\texttt{read.file()} function where it checks to see if a cleaned
version of the data set already exists, to which that particular file
would take precedence and be used. If no clean version currently exists,
the function goes through the cleaning process and saves the cleaned
data set to the same directory as original data set. For unbiased
samples of the data frame, a seed was set for reproducibility and 200
random observations were drawn without replacement.

\vspace{0.25cm}

The documented R code can be found in section \ref{sec:r-setup}. Key
findings and visualizations of summary statistics were discussed
previously throughout sections \ref{sec:rq} through \ref{sec:findings}.
Based on the points made in this section, its clear that data provenance
was kept in mind for making sure there was a traceable history in the
usage of this data set, whether to resolve potential issues or to cut
down on access times.

\newpage

\subsubsection{Data Collection Handout}
\label{sec:appendix-data-handout}

\begin{figure}[!ht]
    \hrule
    \caption{ \textbf{Handout Page 1} }
    \begin{center}
        \scalebox{1.00}{    \includegraphics[trim = 0 0 0 0,clip,width=0.85\textwidth]{pdfs/handout1.pdf} }
    \end{center}
    \label{fig:handout-1}
    \hrule
\end{figure}

\newpage

\begin{figure}[!ht]
    \hrule
    \caption{ \textbf{Handout Page 2} }
    \begin{center}
        \scalebox{1.00}{    \includegraphics[trim = 0 0 0 0,clip,width=0.85\textwidth]{pdfs/handout2.pdf} }
    \end{center}
    \label{fig:handout-2}
    \hrule
\end{figure}

\newpage

\subsection{Data Set Explained}
\label{sec:appendix-dataset-ex}

In addition to how the data set was described in Section \ref{sec:data},
explanations regarding each attribute can be found below. After the
collaborative effort of all students having their data compiled, the
data set ended up having 428 total observations. However, the data set
was filled with an enormous amount of NA values in the body
measurements, potentially due to the time constraints of some students
or lack of attempt to fill out all fields. As a result, running the
function \texttt{complete.cases()} on the data set during the data
cleaning process returned a data frame containing only 262 observations,
about 61.21\% of the original data set size. \texttt{complete.cases()}
was used over \texttt{na.omit()} because the latter function would omit
all rows containing NA values based on all columns, including the rows
where the non-body measurement attributes had NA values, whereas the
former function would omit NA values only for the specified body
measurement columns.

\begin{figure}[!ht]
    \caption{ \textbf{Description of Each Field} }
    \begin{center}
        \scalebox{1.00}{    \includegraphics[trim = 0 350 100 35,clip,width=0.85\textwidth]{pdfs/datasets_explained.pdf} }
    \end{center}
    \label{fig:datasets_explained}
\end{figure}

\newpage

\subsection{R Code Used for Research}
\label{sec:r-setup}

\subsubsection{Sourced Functions}
\label{sec:functions}

\begin{Shaded}
\begin{Highlighting}[]
\NormalTok{prepareMeasureData =}\StringTok{ }\ControlFlowTok{function}\NormalTok{(measure,scale)\{}
  \CommentTok{\# Cleaning: omit NA rows based measured values, not on $side}
\NormalTok{  measure =}\StringTok{ }\NormalTok{measure[}\KeywordTok{complete.cases}\NormalTok{(measure[,}\DecValTok{4}\OperatorTok{:}\DecValTok{26}\NormalTok{]),]}
  
  \CommentTok{\# Cleaning: consistent naming convention}
\NormalTok{  measure}\OperatorTok{$}\NormalTok{gender =}\StringTok{ }\KeywordTok{factor}\NormalTok{(}\KeywordTok{tolower}\NormalTok{(measure}\OperatorTok{$}\NormalTok{gender))}
\NormalTok{  measure}\OperatorTok{$}\NormalTok{gender[measure}\OperatorTok{$}\NormalTok{gender}\OperatorTok{==}\StringTok{"f"}\NormalTok{] =}\StringTok{ "female"}
\NormalTok{  measure}\OperatorTok{$}\NormalTok{gender[measure}\OperatorTok{$}\NormalTok{gender}\OperatorTok{==}\StringTok{"m"}\NormalTok{] =}\StringTok{ "male"}
\NormalTok{  measure}\OperatorTok{$}\NormalTok{units =}\StringTok{ }\KeywordTok{factor}\NormalTok{(}\KeywordTok{tolower}\NormalTok{(measure}\OperatorTok{$}\NormalTok{units))}
\NormalTok{  measure}\OperatorTok{$}\NormalTok{units[measure}\OperatorTok{$}\NormalTok{units}\OperatorTok{==}\StringTok{"inches"}\NormalTok{] =}\StringTok{ "in"}
\NormalTok{  measure}\OperatorTok{$}\NormalTok{units[measure}\OperatorTok{$}\NormalTok{units}\OperatorTok{==}\StringTok{"inch"}\NormalTok{] =}\StringTok{ "in"}
\NormalTok{  measure}\OperatorTok{$}\NormalTok{units[measure}\OperatorTok{$}\NormalTok{units}\OperatorTok{==}\StringTok{"}\CharTok{\textbackslash{}"}\StringTok{in}\CharTok{\textbackslash{}"}\StringTok{"}\NormalTok{] =}\StringTok{ "in"}
\NormalTok{  measure}\OperatorTok{$}\NormalTok{units[measure}\OperatorTok{$}\NormalTok{units}\OperatorTok{==}\StringTok{"cm"}\NormalTok{] =}\StringTok{ "in"}
  
  \CommentTok{\# Converting cm to inches}
  \ControlFlowTok{for}\NormalTok{(row }\ControlFlowTok{in} \DecValTok{1}\OperatorTok{:}\KeywordTok{nrow}\NormalTok{(measure))\{}
    \ControlFlowTok{if}\NormalTok{(measure[row,]}\OperatorTok{$}\NormalTok{units}\OperatorTok{==}\StringTok{"cm"}\NormalTok{)\{}
\NormalTok{      measure[row,}\DecValTok{4}\OperatorTok{:}\DecValTok{26}\NormalTok{] \textless{}{-}}\StringTok{ }\NormalTok{measure[row,}\DecValTok{4}\OperatorTok{:}\DecValTok{26}\NormalTok{]}\OperatorTok{/}\FloatTok{2.54}
\NormalTok{    \}}
\NormalTok{  \}}
  
  \CommentTok{\# Scale data if scale = TRUE}
  \ControlFlowTok{if}\NormalTok{(scale)\{}
\NormalTok{    measure[,}\DecValTok{4}\OperatorTok{:}\DecValTok{26}\NormalTok{] \textless{}{-}}\StringTok{ }\KeywordTok{scale}\NormalTok{(measure[,}\DecValTok{4}\OperatorTok{:}\DecValTok{26}\NormalTok{])}
    \KeywordTok{return}\NormalTok{(measure)}
\NormalTok{  \} }\ControlFlowTok{else}\NormalTok{\{ }\CommentTok{\# If false, return without scaling}
    \KeywordTok{return}\NormalTok{(measure)}
\NormalTok{  \}}
\NormalTok{\}}


\NormalTok{read.file =}\StringTok{ }\ControlFlowTok{function}\NormalTok{(path,scale)\{}
  \KeywordTok{tryCatch}\NormalTok{(}
    \DataTypeTok{expr =}\NormalTok{ \{}
      \CommentTok{\# Open cleaned file if already available}
\NormalTok{      measure =}\StringTok{ }\NormalTok{utils}\OperatorTok{::}\KeywordTok{read.csv}\NormalTok{(}\KeywordTok{paste0}\NormalTok{(path.to.secret,}\StringTok{"measure{-}clean.txt"}\NormalTok{),}
                                \DataTypeTok{header =} \OtherTok{TRUE}\NormalTok{, }\DataTypeTok{quote =} \StringTok{""}\NormalTok{, }\DataTypeTok{sep =} \StringTok{"|"}\NormalTok{)}
      \KeywordTok{return}\NormalTok{(measure)}
\NormalTok{    \},}
    \DataTypeTok{warning =} \ControlFlowTok{function}\NormalTok{(w)\{}
      \CommentTok{\# If no clean file, open original file}
\NormalTok{      measure =}\StringTok{ }\NormalTok{utils}\OperatorTok{::}\KeywordTok{read.csv}\NormalTok{(}\KeywordTok{paste0}\NormalTok{(path.to.secret,}\StringTok{"measure{-}students.txt"}\NormalTok{),}
                                \DataTypeTok{header =} \OtherTok{TRUE}\NormalTok{, }\DataTypeTok{quote =} \StringTok{""}\NormalTok{, }\DataTypeTok{sep =} \StringTok{"|"}\NormalTok{)}
      \CommentTok{\# Clean data}
\NormalTok{      measure =}\StringTok{ }\KeywordTok{prepareMeasureData}\NormalTok{(measure,scale)}
      
      \CommentTok{\# Save cleaned data for later}
      \KeywordTok{write.table}\NormalTok{(measure,}\KeywordTok{paste0}\NormalTok{(path.to.secret,}\StringTok{"measure{-}clean.txt"}\NormalTok{),}\DataTypeTok{sep=}\StringTok{"|"}\NormalTok{,}\DataTypeTok{quote=}\OtherTok{FALSE}\NormalTok{)}
      \KeywordTok{return}\NormalTok{(measure)}
\NormalTok{    \}}
\NormalTok{  )    }
\NormalTok{\}}
\end{Highlighting}
\end{Shaded}

\subsubsection{Libraries, Sourcing Functions, Isolating Sexes}
\label{sec:sourcing}

\begin{Shaded}
\begin{Highlighting}[]
\CommentTok{\# Import necessary libraries}
\KeywordTok{library}\NormalTok{(stats) }\CommentTok{\# For cor()}
\KeywordTok{library}\NormalTok{(devtools) }\CommentTok{\# For source\_url()}
\KeywordTok{library}\NormalTok{(humanVerseWSU)}
\KeywordTok{library}\NormalTok{(Hmisc)}

\CommentTok{\# Source cleaning function}
\NormalTok{path.hub =}\StringTok{ "https://raw.githubusercontent.com/KevnBlack/WSU\_STATS419\_FALL2020/"}
\KeywordTok{source\_url}\NormalTok{(}\KeywordTok{paste0}\NormalTok{(path.hub,}\StringTok{"master/functions/functions{-}project{-}measure.R"}\NormalTok{))}

\NormalTok{path.to.secret =}\StringTok{ "D:/School/Fall 2020/STAT 419/datasets/"}
\NormalTok{measure.df =}\StringTok{ }\KeywordTok{read.file}\NormalTok{(path.to.secret,}\OtherTok{FALSE}\NormalTok{) }\CommentTok{\# Import data without scaling}
\KeywordTok{set.seed}\NormalTok{(}\DecValTok{1}\NormalTok{) }\CommentTok{\# Sample 200 observations from data frame}
\NormalTok{measure.sample =}\StringTok{ }\NormalTok{measure.df[}\KeywordTok{sample}\NormalTok{(}\KeywordTok{nrow}\NormalTok{(measure.df),}\DecValTok{200}\NormalTok{),]}

\CommentTok{\# Isolate male and female data}
\NormalTok{measure.df.m =}\StringTok{ }\NormalTok{measure.sample[measure.sample}\OperatorTok{$}\NormalTok{gender }\OperatorTok{==}\StringTok{ "male"}\NormalTok{,]}
\NormalTok{measure.df.f =}\StringTok{ }\NormalTok{measure.sample[measure.sample}\OperatorTok{$}\NormalTok{gender }\OperatorTok{==}\StringTok{ "female"}\NormalTok{,]}
\end{Highlighting}
\end{Shaded}

\subsubsection{First Research Subquestion}
\label{sec:first-subq}

\begin{Shaded}
\begin{Highlighting}[]
\CommentTok{\# Checking for normality}
\KeywordTok{shapiro.test}\NormalTok{(measure.df.m}\OperatorTok{$}\NormalTok{height.NA)}
\KeywordTok{shapiro.test}\NormalTok{(measure.df.m}\OperatorTok{$}\NormalTok{arm.span.NA)}
\KeywordTok{shapiro.test}\NormalTok{(measure.df.f}\OperatorTok{$}\NormalTok{height.NA)}
\KeywordTok{shapiro.test}\NormalTok{(measure.df.f}\OperatorTok{$}\NormalTok{arm.span.NA)}

\CommentTok{\# Correlation values between height and arm span}
\NormalTok{cor.m.has =}\StringTok{ }\KeywordTok{cor.test}\NormalTok{(measure.df.m}\OperatorTok{$}\NormalTok{height.NA, measure.df.m}\OperatorTok{$}\NormalTok{arm.span.NA)}
\NormalTok{cor.f.has =}\StringTok{ }\KeywordTok{cor.test}\NormalTok{(measure.df.f}\OperatorTok{$}\NormalTok{height.NA, measure.df.f}\OperatorTok{$}\NormalTok{arm.span.NA)}

\CommentTok{\# Graphs with correlation values and trend lines, for males and females}
\KeywordTok{par}\NormalTok{(}\DataTypeTok{mfrow =} \KeywordTok{c}\NormalTok{(}\DecValTok{1}\NormalTok{,}\DecValTok{2}\NormalTok{))}
\KeywordTok{plot}\NormalTok{(measure.df.m}\OperatorTok{$}\NormalTok{height.NA, measure.df.m}\OperatorTok{$}\NormalTok{arm.span.NA,}
     \DataTypeTok{main =} \KeywordTok{paste}\NormalTok{(}\StringTok{"Correlation ="}\NormalTok{, }\KeywordTok{round}\NormalTok{(cor.m.has}\OperatorTok{$}\NormalTok{estimate,}\DecValTok{5}\NormalTok{)), }\DataTypeTok{pch =} \DecValTok{4}\NormalTok{,}
     \DataTypeTok{xlab =} \StringTok{"Male Height (in)"}\NormalTok{, }\DataTypeTok{ylab =} \StringTok{"Male Arm Span (in)"}\NormalTok{, }\DataTypeTok{col =} \StringTok{"blue"}\NormalTok{)}
\KeywordTok{abline}\NormalTok{(}\KeywordTok{lm}\NormalTok{(measure.df.m}\OperatorTok{$}\NormalTok{height.NA }\OperatorTok{\textasciitilde{}}\StringTok{ }\NormalTok{measure.df.m}\OperatorTok{$}\NormalTok{arm.span.NA))}

\KeywordTok{plot}\NormalTok{(measure.df.f}\OperatorTok{$}\NormalTok{height.NA, measure.df.f}\OperatorTok{$}\NormalTok{arm.span.NA,}
     \DataTypeTok{main =} \KeywordTok{paste}\NormalTok{(}\StringTok{"Correlation ="}\NormalTok{, }\KeywordTok{round}\NormalTok{(cor.f.has}\OperatorTok{$}\NormalTok{estimate,}\DecValTok{5}\NormalTok{)), }\DataTypeTok{pch =} \DecValTok{0}\NormalTok{,}
     \DataTypeTok{xlab =} \StringTok{"Female Height (in)"}\NormalTok{, }\DataTypeTok{ylab =} \StringTok{"Female Arm Span (in)"}\NormalTok{, }\DataTypeTok{col =} \StringTok{"red"}\NormalTok{)}
\KeywordTok{abline}\NormalTok{(}\KeywordTok{lm}\NormalTok{(measure.df.f}\OperatorTok{$}\NormalTok{height.NA }\OperatorTok{\textasciitilde{}}\StringTok{ }\NormalTok{measure.df.f}\OperatorTok{$}\NormalTok{arm.span.NA))}
\end{Highlighting}
\end{Shaded}

\subsubsection{Second Research Subquestion}
\label{sec:second-subq}

\begin{Shaded}
\begin{Highlighting}[]
\CommentTok{\# Checking for normality}
\KeywordTok{shapiro.test}\NormalTok{(measure.df.m}\OperatorTok{$}\NormalTok{head.height.NA) }
\KeywordTok{shapiro.test}\NormalTok{(measure.df.m}\OperatorTok{$}\NormalTok{height.NA)}
\KeywordTok{shapiro.test}\NormalTok{(measure.df.f}\OperatorTok{$}\NormalTok{head.height.NA)}
\KeywordTok{shapiro.test}\NormalTok{(measure.df.f}\OperatorTok{$}\NormalTok{height.NA)}

\CommentTok{\# T.test to compare means of populations}
\KeywordTok{t.test}\NormalTok{(measure.df.m}\OperatorTok{$}\NormalTok{head.height.NA, measure.df.f}\OperatorTok{$}\NormalTok{head.height.NA)}
\KeywordTok{t.test}\NormalTok{(measure.df.m}\OperatorTok{$}\NormalTok{height.NA, measure.df.f}\OperatorTok{$}\NormalTok{height.NA)}

\CommentTok{\#\# How many \textquotesingle{}heads\textquotesingle{} on average is a male relative to their \textquotesingle{}height\textquotesingle{}?}
\KeywordTok{mean}\NormalTok{(measure.df.m}\OperatorTok{$}\NormalTok{height.NA)}\OperatorTok{/}\KeywordTok{mean}\NormalTok{(measure.df.m}\OperatorTok{$}\NormalTok{head.height.NA)}
\CommentTok{\#\# How many \textquotesingle{}heads\textquotesingle{} on average is a female relative to their \textquotesingle{}height\textquotesingle{}?}
\KeywordTok{mean}\NormalTok{(measure.df.f}\OperatorTok{$}\NormalTok{height.NA)}\OperatorTok{/}\KeywordTok{mean}\NormalTok{(measure.df.f}\OperatorTok{$}\NormalTok{head.height.NA)}
\end{Highlighting}
\end{Shaded}

\newpage

\begin{Shaded}
\begin{Highlighting}[]
\KeywordTok{library}\NormalTok{(devtools);       }\CommentTok{\# required for source\_url}
\KeywordTok{library}\NormalTok{(Hmisc)}
\KeywordTok{library}\NormalTok{(humanVerseWSU)}

\NormalTok{path.humanVerseWSU =}\StringTok{ "https://raw.githubusercontent.com/MonteShaffer/humanVerseWSU/"}
\KeywordTok{source\_url}\NormalTok{( }\KeywordTok{paste0}\NormalTok{(path.humanVerseWSU,}\StringTok{"master/misc/functions{-}project{-}measure.R"}\NormalTok{) );}

\NormalTok{path.project =}\StringTok{ "D:/School/Fall 2020/STAT 419/git/WSU\_STATS419\_FALL2020/PROJECT{-}01/"}\NormalTok{;}
\NormalTok{path.tables =}\StringTok{ }\KeywordTok{paste0}\NormalTok{(path.project,}\StringTok{"tables/"}\NormalTok{);}
  \KeywordTok{createDirRecursive}\NormalTok{(path.tables);}

\NormalTok{file.correlation =}\StringTok{ }\KeywordTok{paste0}\NormalTok{(path.tables,}\StringTok{"tree{-}correlation{-}table.tex"}\NormalTok{);}

\NormalTok{myData =}\StringTok{ }\KeywordTok{as.matrix}\NormalTok{(trees);  }\CommentTok{\# numeric values only, only what will appear in table}

\CommentTok{\# https://www.overleaf.com/read/srzhrcryjpwn}
\CommentTok{\# keepaspectratio of include graphics }
\CommentTok{\# could scale \textbackslash{}input if still too big ...}
\CommentTok{\# https://tex.stackexchange.com/questions/13460/scalebox{-}knowing{-}how{-}much{-}it{-}scales\#13487}
\KeywordTok{buildLatexCorrelationTable}\NormalTok{(myData, }
  \DataTypeTok{rotateTable =} \OtherTok{TRUE}\NormalTok{,}
  \DataTypeTok{width.table =} \FloatTok{0.60}\NormalTok{, }\CommentTok{\# best for given data ... 0.95 when rotateTable = FALSE}
                      \CommentTok{\# 0.60 when rotateTable = TRUE}
  \DataTypeTok{myFile =}\NormalTok{ file.correlation,}
  \DataTypeTok{myNames =} \KeywordTok{c}\NormalTok{(}\StringTok{"Diameter (in)"}\NormalTok{, }\StringTok{"Height (ft)"}\NormalTok{, }\StringTok{"Volume (ft$\^{}3$)"}\NormalTok{) );}


\KeywordTok{Sys.sleep}\NormalTok{(}\DecValTok{2}\NormalTok{); }\CommentTok{\# in case Knit{-}PDF doesn\textquotesingle{}t like that I just created the file...}
\end{Highlighting}
\end{Shaded}

\newpage

\input{tables/tree-correlation-table}

\newpage




%% appendices go here!


\newpage
\theendnotes

%%%%%%%%%%%%%%%%%%%%%%%%%%%%%%%%%%%  biblio %%%%%%%%
\newpage
\begin{auxmulticols}{2}
\singlespacing 
\bibliography{./../biblio/master.bib}

%%%%%%%%%%%%%%%%%%%%%%%%%%%%%%%%%%%  biblio %%%%%%%%
\end{auxmulticols}

\newpage
{
\hypersetup{linkcolor=black}
\setcounter{tocdepth}{3}
\tableofcontents
}



\end{document}