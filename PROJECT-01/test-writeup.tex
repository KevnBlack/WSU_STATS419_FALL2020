\documentclass[]{article}
\usepackage[left=1in,top=1in,right=1in,bottom=1in]{geometry}


%%%% more monte %%%%
% thispagestyle{empty}
% https://stackoverflow.com/questions/2166557/how-to-hide-the-page-number-in-latex-on-first-page-of-a-chapter
\usepackage{color}
% \usepackage[table]{xcolor} % are they using color?

% \definecolor{WSU.crimson}{HTML}{981e32}
% \definecolor{WSU.gray}{HTML}{5e6a71}

% \definecolor{shadecolor}{RGB}{248,248,248}
\definecolor{WSU.crimson}{RGB}{152,30,50} % use http://colors.mshaffer.com to convert from 981e32
\definecolor{WSU.gray}{RGB}{94,106,113}

%%%%%%%%%%%%%%%%%%%%%%%%%%%%

\newcommand*{\authorfont}{\fontfamily{phv}\selectfont}
\usepackage{lmodern}


  \usepackage[T1]{fontenc}
  \usepackage[utf8]{inputenc}




\usepackage{abstract}
\renewcommand{\abstractname}{}    % clear the title
\renewcommand{\absnamepos}{empty} % originally center

\renewenvironment{abstract}
 {{%
    \setlength{\leftmargin}{0mm}
    \setlength{\rightmargin}{\leftmargin}%
  }%
  \relax}
 {\endlist}

\makeatletter
\def\@maketitle{%
  \pagestyle{empty}
  \newpage
%  \null
%  \vskip 2em%
%  \begin{center}%
  \let \footnote \thanks
    {\fontsize{18}{20}\selectfont\raggedright  \setlength{\parindent}{0pt} \@title \par}%
}
%\fi
\makeatother






\usepackage{color}
\usepackage{fancyvrb}
\newcommand{\VerbBar}{|}
\newcommand{\VERB}{\Verb[commandchars=\\\{\}]}
\DefineVerbatimEnvironment{Highlighting}{Verbatim}{commandchars=\\\{\}}
% Add ',fontsize=\small' for more characters per line
\usepackage{framed}
\definecolor{shadecolor}{RGB}{248,248,248}
\newenvironment{Shaded}{\begin{snugshade}}{\end{snugshade}}
\newcommand{\AlertTok}[1]{\textcolor[rgb]{0.94,0.16,0.16}{#1}}
\newcommand{\AnnotationTok}[1]{\textcolor[rgb]{0.56,0.35,0.01}{\textbf{\textit{#1}}}}
\newcommand{\AttributeTok}[1]{\textcolor[rgb]{0.77,0.63,0.00}{#1}}
\newcommand{\BaseNTok}[1]{\textcolor[rgb]{0.00,0.00,0.81}{#1}}
\newcommand{\BuiltInTok}[1]{#1}
\newcommand{\CharTok}[1]{\textcolor[rgb]{0.31,0.60,0.02}{#1}}
\newcommand{\CommentTok}[1]{\textcolor[rgb]{0.56,0.35,0.01}{\textit{#1}}}
\newcommand{\CommentVarTok}[1]{\textcolor[rgb]{0.56,0.35,0.01}{\textbf{\textit{#1}}}}
\newcommand{\ConstantTok}[1]{\textcolor[rgb]{0.00,0.00,0.00}{#1}}
\newcommand{\ControlFlowTok}[1]{\textcolor[rgb]{0.13,0.29,0.53}{\textbf{#1}}}
\newcommand{\DataTypeTok}[1]{\textcolor[rgb]{0.13,0.29,0.53}{#1}}
\newcommand{\DecValTok}[1]{\textcolor[rgb]{0.00,0.00,0.81}{#1}}
\newcommand{\DocumentationTok}[1]{\textcolor[rgb]{0.56,0.35,0.01}{\textbf{\textit{#1}}}}
\newcommand{\ErrorTok}[1]{\textcolor[rgb]{0.64,0.00,0.00}{\textbf{#1}}}
\newcommand{\ExtensionTok}[1]{#1}
\newcommand{\FloatTok}[1]{\textcolor[rgb]{0.00,0.00,0.81}{#1}}
\newcommand{\FunctionTok}[1]{\textcolor[rgb]{0.00,0.00,0.00}{#1}}
\newcommand{\ImportTok}[1]{#1}
\newcommand{\InformationTok}[1]{\textcolor[rgb]{0.56,0.35,0.01}{\textbf{\textit{#1}}}}
\newcommand{\KeywordTok}[1]{\textcolor[rgb]{0.13,0.29,0.53}{\textbf{#1}}}
\newcommand{\NormalTok}[1]{#1}
\newcommand{\OperatorTok}[1]{\textcolor[rgb]{0.81,0.36,0.00}{\textbf{#1}}}
\newcommand{\OtherTok}[1]{\textcolor[rgb]{0.56,0.35,0.01}{#1}}
\newcommand{\PreprocessorTok}[1]{\textcolor[rgb]{0.56,0.35,0.01}{\textit{#1}}}
\newcommand{\RegionMarkerTok}[1]{#1}
\newcommand{\SpecialCharTok}[1]{\textcolor[rgb]{0.00,0.00,0.00}{#1}}
\newcommand{\SpecialStringTok}[1]{\textcolor[rgb]{0.31,0.60,0.02}{#1}}
\newcommand{\StringTok}[1]{\textcolor[rgb]{0.31,0.60,0.02}{#1}}
\newcommand{\VariableTok}[1]{\textcolor[rgb]{0.00,0.00,0.00}{#1}}
\newcommand{\VerbatimStringTok}[1]{\textcolor[rgb]{0.31,0.60,0.02}{#1}}
\newcommand{\WarningTok}[1]{\textcolor[rgb]{0.56,0.35,0.01}{\textbf{\textit{#1}}}}



\title{\textbf{\textcolor{WSU.crimson}{A boring (academic) title or a clever title?}} \newline \textbf{\textcolor{WSU.gray}{A secondary title}}  }

%  

% \author{ \Large true \hfill \normalsize \emph{} }
\author{\Large YOUR NAME HERE\vspace{0.05in} \newline\normalsize\emph{Washington State University}  }


\date{November 02, 2020}
\setcounter{secnumdepth}{3}

\usepackage{titlesec}
% See the link above: KOMA classes are not compatible with titlesec any more. Sorry.
% https://github.com/jbezos/titlesec/issues/11
\titleformat*{\section}{\bfseries}
\titleformat*{\subsection}{\bfseries\itshape}
\titleformat*{\subsubsection}{\itshape}
\titleformat*{\paragraph}{\itshape}
\titleformat*{\subparagraph}{\itshape}

% https://code.usgs.gov/usgs/norock/irvine_k/ip-092225/


%\titleformat*{\section}{\normalsize\bfseries}
%\titleformat*{\subsection}{\normalsize\itshape}
%\titleformat*{\subsubsection}{\normalsize\itshape}
%\titleformat*{\paragraph}{\normalsize\itshape}
%\titleformat*{\subparagraph}{\normalsize\itshape}

% https://tex.stackexchange.com/questions/233866/one-column-multicol-environment#233904
\usepackage{environ}
\NewEnviron{auxmulticols}[1]{%
  \ifnum#1<2\relax% Fewer than 2 columns
    %\vspace{-\baselineskip}% Possible vertical correction
    \BODY
  \else% More than 1 column
    \begin{multicols}{#1}
      \BODY
    \end{multicols}%
  \fi
}





\usepackage{natbib}
\setcitestyle{aysep={}} %% no year, comma just year
% \usepackage[numbers]{natbib}
\bibliographystyle{./../biblio/ormsv080.bst}



\usepackage[strings]{underscore} % protect underscores in most circumstances




\newtheorem{hypothesis}{Hypothesis}
\usepackage{setspace}


%%%%%%%%%%%%%%%%%%%%%%%%%%%%%%%%%%%%%%%%%%%%%%%%%%%%%
%%% MONTE ADDS %%%

\usepackage{fancyhdr} % fancy header 
\usepackage{lastpage} % last page 

\usepackage{multicol}


\usepackage{etoolbox}
\AtBeginEnvironment{quote}{\singlespacing\small}
% https://tex.stackexchange.com/questions/325695/how-to-style-blockquote


\usepackage{soul}			%% allows strike-through
\usepackage{url}			%% fixes underscores in urls
\usepackage{csquotes}		%% allows \textquote in references
\usepackage{rotating}		%% allows table and box rotation
\usepackage{caption}		%% customize caption information
\usepackage{booktabs}		%% enhance table/tabular environment
\usepackage{tabularx}		%% width attributes updates tabular
\usepackage{enumerate}		%% special item environment
\usepackage{enumitem}		%% special item environment

\usepackage{lineno}		%% allows linenumbers for editing using \linenumbers
\usepackage{hanging}


\usepackage{mathtools}  	%% also loads amsmath
\usepackage{bm}		%% bold-math
\usepackage{scalerel}	%% scale one element (make one beta bigger font)

\newcommand{\gFrac}[2]{ \genfrac{}{}{0pt}{1}{{#1}}{#2} }

\newcommand{\betaSH}[3]{  \gFrac{\text{\tiny #1}}{{\text{\tiny #2}}}\hat{\beta}_{\text{#3}}   }
\newcommand{\betaSB}[3]{              ^{\text{#1}} _{\text{#2}} \bm{\beta} _{\text{#3}}                   }  %% bold
\newcommand{\bigEQ}{  \scaleobj{1.5}{{\ }= } }
\newcommand{\bigP}[1]{  \scaleobj{1.5}{#1 } }





\usepackage{endnotes}  % he already does this ...
\renewcommand{\enotesize}{\normalsize}
% https://tex.stackexchange.com/questions/99984/endnotes-do-not-be-superscript-and-add-a-space
\renewcommand\makeenmark{\textsuperscript{[\theenmark]}} % in brackets %
% https://tex.stackexchange.com/questions/31574/how-to-control-the-indent-in-endnotes
\patchcmd{\enoteformat}{1.8em}{0pt}{}{}

\patchcmd{\theendnotes}
  {\makeatletter}
  {\makeatletter\renewcommand\makeenmark{\textbf{[\theenmark]} }}
  {}{}



% https://tex.stackexchange.com/questions/141906/configuring-footnote-position-and-spacing

\addtolength{\footnotesep}{5mm} % change to 1mm

\renewcommand{\thefootnote}{\textbf{\arabic{footnote}}}
\let\footnote=\endnote
%\renewcommand*{\theendnote}{\alph{endnote}}
%\renewcommand{\theendnote}{\textbf{\arabic{endnote}}}


\renewcommand*{\notesname}{ENDNOTES}

\makeatletter
\def\enoteheading{\section*{\notesname
  \@mkboth{\MakeUppercase{\notesname}}{\MakeUppercase{\notesname}}}%
  \mbox{}\par\vskip-2.3\baselineskip\noindent\rule{.5\textwidth}{0.4pt}\par\vskip\baselineskip}
\makeatother


\renewcommand*{\contentsname}{TABLE OF CONTENTS}

\renewcommand*{\refname}{REFERENCES}


%\usepackage{subfigure}
\usepackage{subcaption}

\captionsetup{labelfont=bf}  % Make Table / Figure bold

%%% you could add elements here ... monte says .... %%%
%\usepackage{mypackageForCapitalH}


%%%%%%%%%%%%%%%%%%%%%%%%%%%%%%%%%%%%%%%%%%%%%%%%%%%%%

% set default figure placement to htbp
\makeatletter
\def\fps@figure{htbp}
\makeatother


% move the hyperref stuff down here, after header-includes, to allow for - \usepackage{hyperref}

\makeatletter
\@ifpackageloaded{hyperref}{}{%
\ifxetex
  \PassOptionsToPackage{hyphens}{url}\usepackage[setpagesize=false, % page size defined by xetex
              unicode=false, % unicode breaks when used with xetex
              xetex]{hyperref}
\else
  \PassOptionsToPackage{hyphens}{url}\usepackage[draft,unicode=true]{hyperref}
\fi
}

\@ifpackageloaded{color}{
    \PassOptionsToPackage{usenames,dvipsnames}{color}
}{%
    \usepackage[usenames,dvipsnames]{color}
}
\makeatother
\hypersetup{breaklinks=true,
            bookmarks=true,
            pdfauthor={YOUR NAME HERE (Washington State University)},
             pdfkeywords = {multiple comparisons to control; multivariate chi-square distribution;
nonlinear growth curves; Richard's curve; simulated critical points},  
            pdftitle={A boring (academic) title or a clever title?: A secondary title},
            colorlinks=true,
            citecolor=blue,
            urlcolor=blue,
            linkcolor=magenta,
            pdfborder={0 0 0}}
\urlstyle{same}  % don't use monospace font for urls

% Add an option for endnotes. -----

%
% add tightlist ----------
\providecommand{\tightlist}{%
\setlength{\itemsep}{0pt}\setlength{\parskip}{0pt}}

% add some other packages ----------

% \usepackage{multicol}
% This should regulate where figures float
% See: https://tex.stackexchange.com/questions/2275/keeping-tables-figures-close-to-where-they-are-mentioned
\usepackage[section]{placeins}



\pagestyle{fancy}   
\lhead{\textcolor{WSU.crimson}{\textbf{ A boring (academic) title or a clever title? }}}
\chead{}
\rhead{\textcolor{WSU.gray}{\textbf{  Page\ \thepage\ of\ \protect\pageref{LastPage} }}}
\lfoot{}
\cfoot{}
\rfoot{}


\begin{document}
	
% \pagenumbering{arabic}% resets `page` counter to 1 
%
% \maketitle

{% \usefont{T1}{pnc}{m}{n}
\setlength{\parindent}{0pt}
\thispagestyle{plain}
{\fontsize{18}{20}\selectfont\raggedright 
\maketitle  % title \par  

}

{
   \vskip 13.5pt\relax \normalsize\fontsize{11}{12} 
   
\textbf{\authorfont YOUR NAME HERE} \hskip 15pt \emph{\small Washington State University}   

}

}








\begin{abstract}

    \hbox{\vrule height .2pt width 39.14pc}

    \vskip 8.5pt % \small 

\noindent In this article we compare the \emph{empirical characteristic function}
\citep{Tukey:1977, Becker:1988} to a
\emph{moment-generating-functional form} to compute the proportion of
hypotheses \(m\) that are rejected under the null hypothesis.
\vspace{0.25in}

\noindent Here is a second paragraph of the abstract (if necessary), and
with the pipe notation it doesn't break. Notice it still needs to be
indented. \vspace{0.25in}

\noindent Generally, we write this abstract last. Often it is called the
executive summary. It should succinctly summarize the entire document.
You can include references such as this one to the Appendices section
\ref{sec:appendix} if necessary.


\vskip 8.5pt \noindent \textbf{\underline{Keywords}:} multiple comparisons to control; multivariate chi-square distribution;
nonlinear growth curves; Richard's curve; simulated critical points \par

    




    
    \hbox{\vrule height .2pt width 39.14pc}
    \vskip 5pt 
    \hfill \textbf{\textcolor{WSU.gray}{ November 02, 2020 } }
    \vskip 5pt 
    
\end{abstract}


\vskip -8.5pt



 % removetitleabstract

\noindent  

\section{Introduction}
\label{sec:intro}

Write something here.

{[}ONE GRAPHIC{]}

{[}TWO GRAPHICS AS ONE{]}

Write something here.

\section{Research Question:  What is my primary question}
\label{sec:rq}

\subsection{What is my secondary question}
\label{sec:rq2}

\subsection{What is my other secondary question}
\label{sec:rq3}

\section{Data Description}
\label{sec:data}

Very brief introduction to the data, how it was collected, and so on.
Remember that everything is covered (who, what, when, where, why, how,
so what, and so on). Reference the section in the Appendix with greater
detail about the data provenance. This section should be about two
paragraphs, and the Appendix should have more information.

\subsection{Summary of Sample}
\label{sec:data-sample}

\subsection{Summary Statistics of Data}
\label{sec:data-summary}

\section{Key Findings}
\label{sec:findings}

\section{Conclusion}
\label{sec:conclusion}

\newpage

This was a new page

This is a newline. \newline  Here is some more text.

Below are some example code that may benefit you in preparing your
document. \newline

\vspace{0.25in}

\noindent Please state your name: \hrulefill \newline I was born on
\hrulefill in \hrulefill \vspace{0.25in}

\begin{equation}
\label{eq:my-model}
    Y_{jt} = \alpha + \bm{\beta}X_{jt} + \upsilon_{j}  + \varepsilon_{jt} ,
\end{equation}

\noindent where \(\alpha\) is the grand mean, \(\upsilon_{j}\) is the
fixed-time country mean, \(X_{jt}\) (country \(j\) at time \(t\)) is the
matrix of country-level observations for the vector of aforementioned
parameters \(\bm{\beta}\), and \(\varepsilon_{jt}\) represents the
residual idiosyncratic disturbance. Our panel data set consists of
repeated observations of countries over time. Therefore, we employ
cross-section time-series models. This approach redefines
Equation\textasciitilde{}\ref{eq:my-model} by subtracting time-demeaned
values. This \emph{within} transformation subtracts constant country
effects for the dependent variable \(\bar{Y_{j}}\), the predictor
variables \(\bar{X_{j}}\), and the intercept \(\bar{\upsilon_{j}}\):

\begin{equation}
\label{eq:my-random}
    (Y_{jt} - \theta \bar{Y_{j}}) = (1-\theta)\alpha + \bm{\beta}(X_{jt} - \bar{X_{j}}) +  (\upsilon_{jt} - \theta \bar{\upsilon_{j}})  ,
\end{equation}

\noindent If \(\theta = 0\), the model reduces to a basic pooled
ordinary-least-squares (OLS) model; if \(\theta = 1\), the model reduces
to a fixed-effects model; otherwise the model represents a
random-effects model. The pooled OLS estimation is biased if country
effects exist \citep{Hsiao:2003}. The random-effects model may be
susceptible to omitted-variable bias \citep{Wooldridge:2006}: bias
because a predictor was excluded from the model specification.
Conversely, the fixed-effects model is not susceptible to this bias as
it captures unobserved intracountry variation around its average
country-level ``fixed effect." Panel-data analysis commonly has issues
with heteroskedasticity, serial autocorrelation, and cross-sectional
autocorrelation.

\vspace{0.5in}

\(i=1\) and \[i = 1\]

\vspace{0.5in}

\begin{tabular}{ c c c c c}
  1 & 2 & 3 & 4 & 5 \\
  \hline
  6 & 7 & 8 & 9 & 10
\end{tabular}

\vspace{0.5in}

See Figure \ref{fig:conceptual-model}.

\newpage

This is a
footnote\footnote{This is a footnote that can be really long.  \newline You can have multiple paragraphs in the footnote.  You can have \underline{underline} or \textbf{bold} or \emph{italics}.  You can even have a math equation inline. \newline In this section, we review the regression results to summarize our findings.  First, we examine each model for significance, and conclude the hypothesized models fit well with the data.  Second, we conclude that the fixed country effects represent consistent and unbiased parameter estimates.  Third, with the use of the \citet{Driscoll:1998} robust standard errors, we adjust any variance bias to ascertain the significance of these consistent estimates.  Therefore, we are able to make inferences about the hypotheses using our model estimates.  For ease of interpretation across these 12 models, we introduce $\betaSH{{ \ \ }M1}{Total}{1}$ as notation to refer to parameter estimate $\hat{\beta}_{1}$ (HDI) for the Total Sample and (M1) Model 1:  Main Effects.  We proceed by reporting findings for the total sample. \newline The footnotes are automatically converted to "endnotes" and will be included at the end of the document.  It will finish when you have that outer brace like this.}
that can be placed within a document.

\vspace{1.5in}

Refer to the Appendices in section\textasciitilde{}\ref{sec:appendix}
where I am going to cite John \citep[pp. 2-3]{Tukey:1962}.

Here is a quote by \citet[pp. 2-3]{Tukey:1962}:

\begin{quote}
For a long time I have thought I was a statistician, interested in inferences from the particular to the general.  But as I have watched mathematical statistics evolve, I have had to cause to wonder and to doubt. [...] All in all, I have come to feel that my central interest is in \emph{data analysis}, which I take to include among other things: procedures for analyzing data, techniques for interpreting the results of such procedures, ways of planning the gathering of data to make its analysis easier, more precise or more accurate, and all the machinery and results of (mathematical) statistics which apply to analyzing the data.

Large parts of data analysis are inferential in the sample-to-population sense, but these are only parts, not the whole.  Large parts of data analysis are incisive, laying bare indications which we could not perceive by simple and direct examination of the raw data, but these too are only parts, not the whole.  Some parts of data analysis, as the term is her stretch beyond its philology, are allocation, in the sense that they guide us in the distribution of effort and other valuable considerations in observation, experimentation, or analysis.  Data analysis is a larger and more varied field than inference, or incisive procedures, or allocation.

Statistics has contributed much to data analysis.  In the future it can, and in my view should, contribute more.  For such contributions to exist, and be valuable, it is not necessary that they be direct.  They need not provide new techniques, or better tables for old techniques, in order to influence the practice of data analysis.
\end{quote}

\newpage


\begin{sidewaystable}[!htbp]
\setlength\tabcolsep{2pt}
\footnotesize
\centering

\caption{\textbf{ Descriptive Statistics and Correlation Analysis } } \label{table:correlation}
\vspace{.5cm}
\begin{tabularx}{0.9\textwidth}{r@{ \ \ }p{5cm}r@{}lr@{}lr@{}lr@{}lr@{}lr@{}lr@{}lr@{}lr@{}lr@{}lr@{}lr@{}lr@{}lp{1cm}}  %% need gutter r/l cspace, and so on ... 
\hline \\[-0.22cm]
\multicolumn{2}{c}{\textbf{ }}  & \multicolumn{2}{p{0.5cm}}{ } & \multicolumn{2}{c}{\textbf{M}} & \multicolumn{2}{c}{\textbf{SD}} & \multicolumn{2}{p{1.5cm}}{ }  & \multicolumn{2}{c}{\textbf{1}}  & \multicolumn{2}{c}{\textbf{2}}  & \multicolumn{2}{c}{\textbf{3}}  & \multicolumn{2}{c}{\textbf{4}}  & \multicolumn{2}{c}{\textbf{5}}  & \multicolumn{2}{c}{\textbf{6}}  & \multicolumn{2}{c}{\textbf{7}}  & \multicolumn{2}{c}{\textbf{8}}  & \multicolumn{2}{c}{\textbf{9}}  \\
\hline
 & \\
1. & ln(High-technology Exports) & \multicolumn{2}{p{0.5cm}}{ }  & 22&.03 & 2&.22 & \multicolumn{2}{p{0.5cm}}{ }   & \multicolumn{2}{c}{1}   & \multicolumn{2}{c}{ \  \  \  \  \  \  \ }   & \multicolumn{2}{c}{ \  \  \  \  \  \  \ }   & \multicolumn{2}{c}{ \  \  \  \  \  \  \ }   & \multicolumn{2}{c}{ \  \  \  \  \  \  \ }   & \multicolumn{2}{c}{ \  \  \  \  \  \  \ }   & \multicolumn{2}{c}{ \  \  \  \  \  \  \ }   & \multicolumn{2}{c}{ \  \  \  \  \  \  \ }   & \multicolumn{2}{c}{ \  \  \  \  \  \  \ }  \\[0.33cm]
2. & Human Development (HDI) & \multicolumn{2}{p{0.5cm}}{ }  & &.78 & &.10 & \multicolumn{2}{p{0.5cm}}{ }  & &.36{$^{***}$}  & \multicolumn{2}{c}{1}   & \multicolumn{2}{c}{ \  \  \  \  \  \  \ }   & \multicolumn{2}{c}{ \  \  \  \  \  \  \ }   & \multicolumn{2}{c}{ \  \  \  \  \  \  \ }   & \multicolumn{2}{c}{ \  \  \  \  \  \  \ }   & \multicolumn{2}{c}{ \  \  \  \  \  \  \ }   & \multicolumn{2}{c}{ \  \  \  \  \  \  \ }   & \multicolumn{2}{c}{ \  \  \  \  \  \  \ }  \\[0.33cm]
3. & Trade Openness (OPEN) & \multicolumn{2}{p{0.5cm}}{ }  & 81&.25 & 60&.73 & \multicolumn{2}{p{0.5cm}}{ }  & &.22{$^{***}$} & &.21{$^{***}$}  & \multicolumn{2}{c}{1}   & \multicolumn{2}{c}{ \  \  \  \  \  \  \ }   & \multicolumn{2}{c}{ \  \  \  \  \  \  \ }   & \multicolumn{2}{c}{ \  \  \  \  \  \  \ }   & \multicolumn{2}{c}{ \  \  \  \  \  \  \ }   & \multicolumn{2}{c}{ \  \  \  \  \  \  \ }   & \multicolumn{2}{c}{ \  \  \  \  \  \  \ }  \\[0.33cm]
4. & WTO & \multicolumn{2}{p{0.5cm}}{ }  & &.82 & &.39 & \multicolumn{2}{p{0.5cm}}{ }  & &.11{$^{**}$} & &.18{$^{***}$} & &.14{$^{***}$}  & \multicolumn{2}{c}{1}   & \multicolumn{2}{c}{ \  \  \  \  \  \  \ }   & \multicolumn{2}{c}{ \  \  \  \  \  \  \ }   & \multicolumn{2}{c}{ \  \  \  \  \  \  \ }   & \multicolumn{2}{c}{ \  \  \  \  \  \  \ }   & \multicolumn{2}{c}{ \  \  \  \  \  \  \ }  \\[0.33cm]
5. & Team: Sole Inventor & \multicolumn{2}{p{0.5cm}}{ }  & &.37 & &.16 & \multicolumn{2}{p{0.5cm}}{ }  & -&.05 & &.04 & -&.22{$^{***}$} & -&.30{$^{***}$}  & \multicolumn{2}{c}{1}   & \multicolumn{2}{c}{ \  \  \  \  \  \  \ }   & \multicolumn{2}{c}{ \  \  \  \  \  \  \ }   & \multicolumn{2}{c}{ \  \  \  \  \  \  \ }   & \multicolumn{2}{c}{ \  \  \  \  \  \  \ }  \\[0.33cm]
6. & Team: Multiregional & \multicolumn{2}{p{0.5cm}}{ }  & &.38 & &.19 & \multicolumn{2}{p{0.5cm}}{ }  & -&.31{$^{***}$} & -&.32{$^{***}$} & &.11{$^{***}$} & &.25{$^{***}$} & -&.72{$^{***}$}  & \multicolumn{2}{c}{1}   & \multicolumn{2}{c}{ \  \  \  \  \  \  \ }   & \multicolumn{2}{c}{ \  \  \  \  \  \  \ }   & \multicolumn{2}{c}{ \  \  \  \  \  \  \ }  \\[0.33cm]
7. & Firm = 0 & \multicolumn{2}{p{0.5cm}}{ }  & &.17 & &.14 & \multicolumn{2}{p{0.5cm}}{ }  & -&.46{$^{***}$} & -&.33{$^{***}$} & -&.23{$^{***}$} & -&.20{$^{***}$} & &.35{$^{***}$} & -&.04  & \multicolumn{2}{c}{1}   & \multicolumn{2}{c}{ \  \  \  \  \  \  \ }   & \multicolumn{2}{c}{ \  \  \  \  \  \  \ }  \\[0.33cm]
8. & Firm = 2+ & \multicolumn{2}{p{0.5cm}}{ }  & &.04 & &.05 & \multicolumn{2}{p{0.5cm}}{ }  & -&.03 & -&.10{$^{**}$} & -&.02 & &.09{$^{**}$} & -&.19{$^{***}$} & &.20{$^{***}$} & -&.03  & \multicolumn{2}{c}{1}   & \multicolumn{2}{c}{ \  \  \  \  \  \  \ }  \\[0.33cm]
9. & Independent Claims (DEPTH) & \multicolumn{2}{p{0.5cm}}{ }  & 2&.43 & &.54 & \multicolumn{2}{p{0.5cm}}{ }  & &.00 & &.04 & &.06{$^{\dagger}$} & &.32{$^{***}$} & -&.38{$^{***}$} & &.35{$^{***}$} & -&.14{$^{***}$} & &.03  & \multicolumn{2}{c}{1}  \\[0.33cm]
10. & Unique Technologies (BREADTH) & \multicolumn{2}{p{0.5cm}}{ }  & 1&.64 & &.19 & \multicolumn{2}{p{0.5cm}}{ }  & &.00 & -&.03 & -&.05{$^{\dagger}$} & -&.20{$^{***}$} & -&.20{$^{***}$} & &.10{$^{**}$} & -&.05 & &.00 & &.14{$^{***}$} \\
 & \\
\hline
\end{tabularx}
\begin{tabularx}{0.9\textwidth}{{r@{}l}}
{\tiny {$^{\dagger}$} } & {\tiny {$ p < .10$} }  \\
{\tiny {$^{*}$} } & {\tiny {$ p < .05$} }  \\
{\tiny {$^{**}$} } & {\tiny {$ p < .01$} }  \\
{\tiny {$^{***}$} } & {\tiny {$ p < .001$} }  \\
%% .8 * .95 = 0.76
%% .9 * .95 = 0.855
\multicolumn{2}{p{\dimexpr .855\linewidth-2\tabcolsep}}{{ \footnotesize \begin{hangparas}{0.5in}{1} Notes: Pearson pairwise correlations are reported. Correlations and Summary Statistics fail to account for country-level and time-varying effects.  We caution the reader to make inferences from these basic associations.  They do not capture joint associations inherent to a regression model.    \end{hangparas}   }} \\[-0.25cm]
\hline
\end{tabularx}

\end{sidewaystable}


\newpage

\newpage
\section{APPENDICES}
\label{sec:appendix}

\subsection{Data Provenance}
\label{sec:appendix-data-provenance}

\newpage
\subsubsection{Data Collection Handout}
\label{sec:appendix-data-handout}

\newpage

\subsection{Preparing the Report Workspace as a subsection}
\label{sec:appendix-setup}

\subsubsection{Preparing the Report Workspace as a subsubsection}
\label{sec:appendix-setup2}

\paragraph{Preparing the Report Workspace as a paragraph}
\label{sec:appendix-setup3}

\subparagraph{Preparing the Report Workspace as a subparagrah}
\label{sec:appendix-setup4}

Below is the necessary functions and libraries required to run the code
referenced in this document.

\begin{Shaded}
\begin{Highlighting}[]
\CommentTok{# library(devtools);       # required for source_url}
\CommentTok{# }
\CommentTok{# path.humanVerseWSU = "https://raw.githubusercontent.com/MonteShaffer/humanVerseWSU/"}
\CommentTok{# source_url( paste0(path.humanVerseWSU,"master/misc/functions-project-measure.R") );}
\end{Highlighting}
\end{Shaded}

Below is the code to load the data and prepare it for analysis.

\begin{Shaded}
\begin{Highlighting}[]
\CommentTok{# path.project = "C:/_git_/WSU_STATS419_FALL2020/project-measure/";}
\CommentTok{# }
\CommentTok{# path.to.secret = "C:/Users/Alexander Nevsky/Dropbox/WSU-419/Fall 2020/__student_access__/_SECRET_/";}
\CommentTok{# }
\CommentTok{# measure = utils::read.csv( paste0(path.to.secret, "measure-students.txt"), header=TRUE, quote="", sep="|");}

\CommentTok{#path.github = "https://raw.githubusercontent.com/this-IS-YOUR-PATH-TO-GITHUB/";}
\CommentTok{#source_url( paste0(path.github,"master/functions/functions-project-measure.R") );}

\CommentTok{# this is your function}
\CommentTok{# put in the same "units"}
\CommentTok{# merge left/right}
\CommentTok{# build proportion data}
\CommentTok{# and so on ... }
\CommentTok{# measure.df = prepareMeasureData(measure);}
\end{Highlighting}
\end{Shaded}




%% appendices go here!


\newpage
\theendnotes

%%%%%%%%%%%%%%%%%%%%%%%%%%%%%%%%%%%  biblio %%%%%%%%
\newpage
\begin{auxmulticols}{2}
\singlespacing 
\bibliography{./../biblio/master.bib}

%%%%%%%%%%%%%%%%%%%%%%%%%%%%%%%%%%%  biblio %%%%%%%%
\end{auxmulticols}

\newpage
{
\hypersetup{linkcolor=black}
\setcounter{tocdepth}{3}
\tableofcontents
}



\end{document}